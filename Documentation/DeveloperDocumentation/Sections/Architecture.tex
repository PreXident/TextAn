% Chapter: contains the system architecture

This chapter contains a detailed description of overall architecture of the client
side, the server side and the communication between them. 

\section{Architecture Overview}

The architecture of \textan{} is based on client-server model. Two main components
are the \textan{} server (see Chapter \ref{ch:ServerArch}) and \textan{} client 
(see Chapter \ref{ch:ClientArch}). They communicate via W3C web services
(SOAP protocol), see Figure \ref{fig:Architecture}.

\begin{figure}[!htb]
        \centering
        \includegraphics[height=16cm]{Images/Architecture}
        \caption{Architecture overview.}
        \label{fig:Architecture}
\end{figure}

\section{Communication}
\label{sec:Communication}

The \textan{} server provides two webservice interfaces for the client to use -
\emph{DataProvider} and \emph{DocumentProcessor}. The
\emph{DataProvider} offers operations for manipulating data in the database,
for example \emph{getObject}, \emph{mergeObjects} and \emph{updateDocument}.
The \emph{DocumentProcessor} offers all operations related to the processing,
such as \emph{getEditingTicket}, \emph{getEntitiesFromString} and
\emph{saveProcessedDocumentFromString}.
For more details, consult the WSDL descriptions and Javadoc of the server
implementation.

The usage of webservice \emph{DataProvider} is straightforward, its operations
are completely independent of each other. On the other hand, the service
\emph{DocumentProcessor} is more demanding. Its operations need to be
called in specific order to achieve proper recognition (see Figure
\ref{fig:ClientServerCommunication}). At the beginning, the client needs to obtain
the ticket which stores vital piece of information about processing for all
following calls. For three recognition operations there are two variants. First
for processing report already stored in the database (called\emph{*ById}) and
second for inserting completely new document to the database (called
\emph{*FromString}). Operations \emph{getEntities*} recognize entities in
given report. Operations \emph{getAssignments*} assign objects from the database
to recognized entities and finally operations \emph{getRelations*} recognize
relations between objects. Please note that \emph{getRelations*} is not
implemented in this version of \textan{} and it returns empty response with no
relations recognized.

The recognition operations do not store anything on the database. For this
purpose, there are three operations. The first two operations are \emph{saveProcessedDocumentFromString} 
and \emph{saveProcessedDocumentById} with similar attributes to 
two variants of the recognition operations.
The third operation \emph{rewriteAndSaveProcessedDocumentById} gets input
parameters from both document id and document text, because it overwrites the report
text stored in the database. The third operation is called when the document is
changed by others while being processed by the user. The user can
force the document into returning back to the original version or provides new one. 
All three operations
take \emph{force} parameter specifying whether the save should happen regardless of
external changes to the database, e.g. new objects, new
relations and newly joined objects.

If the \emph{force} parameter is \emph{false} and such external changes have
been detected, save operation returns \emph{false} and method \emph{getProblems} may be
called to get information about the changes. After making adjustments to
recognizing/assignments if needed, save operation can be called with
\emph{force} parameter set to \emph{true}.

During the process of a document already stored in the database, other users
might alter the document in the database. If this happens, the subsequent call of processing
operation returns fault \emph{documentChangedException} as a warning. If the
user wants to overwrite these changes as mentioned above, the client should
switch to \emph{*FromString} recognizing operations and save the report by
operation \emph{rewriteAndSaveProcessedDocumentById}.

If the document stored in the database is processed by another user while
being processed by a local user, the subsequent call of processing operation
returns fault \emph{documentAlreadyProcessedException} and the processing must
end as no report can be processed multiple times.

\begin{figure}[!htb]
        \centering
        \includegraphics{Images/ClientServerCommunication}
        \caption{Example of Client-Server communication during report processing.}
        \label{fig:ClientServerCommunication}
\end{figure}
