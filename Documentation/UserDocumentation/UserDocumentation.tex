%Use Lualatex for interpretation!
% lualatex -interaction=nonstopmode --shell-escape UserDocumentation.tex

%You have to have those fonts in your system:
% - Calibri (standard windows font)
% - Trebuchet MS (standard windows font)
% - GROTESKIA (http://www.dafont.com/groteskia.font)

\documentclass[12pt,a4paper]{report}

%\usepackage[utf8]{inputenc} % for old latex interpreters
%\usepackage[english]{babel}

\usepackage{polyglossia}
\setdefaultlanguage[variant=american]{english} % or ?british
\setotherlanguage{czech}

\usepackage{geometry}
\usepackage[cmyk]{xcolor}
\usepackage{amsmath}
\usepackage[some]{background}
\usepackage{listings}
\usepackage{footmisc}
\usepackage{titlesec}
\usepackage{fontspec}
\usepackage{epstopdf,epsfig} 

\definecolor{TextanDarkRed}{cmyk}{.37,.94,.83,.59}
\definecolor{TextanRed}{cmyk}{.0,.879,.844,.220}
\definecolor{javagreen}{rgb}{0.25,0.5,0.35}

\newfontfamily\chapterfont{GROTESKIA}
\newfontfamily\sectionfont{Trebuchet MS}
\titleformat{\chapter}{\chapterfont\fontsize{36pt}{1pt}\selectfont\color{TextanRed}}
  {\thechapter}{20pt}{\chapterfont\fontsize{36pt}{1pt}\selectfont\MakeUppercase}
\titlespacing{\chapter}{0pt}{0pt}{20pt}
\titleformat*{\section}{\LARGE\sectionfont\color{TextanDarkRed}}
\titleformat*{\subsection}{\fontsize{14pt}{1pt}\selectfont\sectionfont\itshape\color{TextanDarkRed}}\titleformat*{\subsubsection}{\sectionfont\color{TextanDarkRed}}

\setmainfont{Calibri}

\backgroundsetup{
scale=1,
angle=0,
opacity=1,
contents={
\begin{tikzpicture}[remember picture,overlay]
  \path [fill=TextanDarkRed] (-0.5\paperwidth,-0.5\paperheight)rectangle (-0.37\paperwidth,0.5\paperheight);
 \end{tikzpicture}}
}

\usepackage[unicode,colorlinks=true]{hyperref}
\hypersetup{pdftitle=TextAn - user documentation}
\hypersetup{pdfauthor={Petr Fanta, Duc Tam Hoang, Adam Huječek, Václav Pernička, Jakub Vlček}}
\hypersetup{linkcolor=black, citecolor=black, urlcolor=black, filecolor=black}

\def\chapwithtoc#1{
\chapter*{#1}
\addcontentsline{toc}{chapter}{#1}
}

% sets numbering for subsubsection
\setcounter{secnumdepth}{3}
\setcounter{tocdepth}{3}

\lstdefinelanguage{properties}{% new language for listings
  basicstyle=\ttfamily,
  sensitive=false,
  morecomment=[l]{\#},      % comment
  morestring=[b]",          % string def
  commentstyle=\color{javagreen},
  basicstyle=\small
}

%Comment macro
%Usage:
% \comment[_assignee_]{_author_}{_comment_}
% \comment{_author_}{_comment_}
\makeatletter
\newcommand{\comment}[3][\@empty]{
  {\color{magenta}[#3 - }
  {\color{green}\ifx\@empty#1\relax Author: #2 \else Assignee: #1; Author: #2\fi}{\color{magenta}]}
}
\makeatother

\newcommand{\textan}{\emph{TextAn}}

\begin{document}

\begin{titlepage}
\BgThispage
\newgeometry{left=4.5cm,top=7cm,bottom=3cm}

\begin{figure}
 \includegraphics{../CommonImages/TEXTAN_logo_grey_B}
\end{figure}
\noindent
\textcolor{TextanRed}{\chapterfont\fontsize{48pt}{1pt}\selectfont\MakeUppercase{User documentation}}\\[15pt]
\textcolor{TextanDarkRed}{\sectionfont\LARGE\MakeUppercase{Version 0.1}}

\vfill
\noindent
\begin{minipage}[b]{.75\textwidth}
\textbf{Authors}\\
Petr Fanta\\
Duc Tam Hoang\\
Adam Huječek\\
Václav Pernička\\
Jakub Vlček
\end{minipage}% This must go next to `\end{minipage}`
\begin{minipage}[b]{.25\textwidth}
\textbf{Supervisor} \\
Ondřej Bojar\\
%\vfill
\\
\\
\textbf{Date}\\
\today
\end{minipage}

\end{titlepage}
\restoregeometry

\pagenumbering{roman}
\tableofcontents

%\chapter*{Intro}
%\addcontentsline{toc}{chapter}{Intro}


\chapter{User Guide}
\pagenumbering{arabic}

\section{Introduction}

% FOR ENTERTAIN: The spider opens his heavy eyes. He stand up, look through the lousy windows. The sun is flaring, kissing all over the earth. He said to himself: ``It's time''. Students are rushing out of the dormitory like rats desert from a sinking ship. Some are going to the public transport, waiting for the regular hooves from the distance. The bus comes and goes, its double tires sing high notes on the road. Some are visiting the car park. The wheels of the cars creaked around, then they crawled into the city centre. A man and girl are crossing street, with their arms around each other's waists. Then the man must have said something supposed to be funny because two of them are laughing like hyenas. The spider retreats from the windows. The daily screen makes him so lonesome and depressed. Another day has just begun.

%TODO what is textan

%TextAn (Text Analyser) is the product of our software project groups (consists of 5 members) at Faculty of Mathematics and Physics, Charles University in Prague (MFF UK). The tool is attributed to Software Project subject, developed in 9 months from Jan 2014 to Sep 2014, and supervised by RNDr. Ondřej Bojar, PhD. It is a client-server tool which support mining structured information from text documents. At the moment, the documents are specified to police report but TextAn could be applied to a wide range of domains. 

% --> to DeveloperDocumentation
% 

%TODO basic usage


\subsection{System overview}

%What is this?

%what textan do, something which is 

%WHAT DOES THE POLICE WANT? OF COURSE NOT ARREST US

% 

% it should be something better, for example, what textan to in a : What textan to. what you can expect in the rest of documentation . If you don't need what we offer in textan, leave a comment and get away.



With the profiliferation of unstructured written texts, the need for a tool to mine the structure out of such documents is on increasing trend. 
\textan{} (Text Analyser) serves such purpose, targeting the Police's reports. 
In other words, \textan{} is a tool which supports mining structured data buried in the Police's reports. The term ``structured data'' consists of two concepts. 
First, it contains name, street, date of birth, crime and other named entities with related to some people. 
Second, the data contain the relations between two entities. 
For example, the relation between a person and his/her name, the relation between two persons. The objective of \textan{} is to provide a robust tool which support the procedure, either automatically and manually.

\textan{} has client-server structure. It supports following operations:
  
  \begin{itemize}
  \item Bring new solution to the classic problem of extracting data from text.
  \item Provide the service for both automatic detection and manual adjustment.
  \item Make the adjustment of data as simple as possible for users.
  \item Make the graphic user interface as fruitful as possible.
  \end{itemize}

This documentation describes \textan{} from the perspective of an user. 
All the concepts are explained in the \emph{Glossary}.

To start using \textan{}, user have to start the client application and log in. 
Once user log in successfully, the main screen of \textan{} is displayed.

\comment{Tam}{What's next?} 

\subsection{Glossary}

\comment{Petr}{definitions of terms in documentation and applications}

\subsection{Conventions}
\comment{Petr}{is this section needed?}

\section{Client Usage}

\subsection{First run}

\subsection{Working space}

\subsection{Settings}
\label{ssec:Settings}

\subsection{Report processing}

\subsection{Listing objects}

\subsection{Graph views}

\subsection{Joining objects}

\chapter{Administrator guide}

\comment{Petr}{client instalation guide in the administrator guide or user guide?}

\section{Server}

\subsection{Installation guide}


This section will show you how to install and build \textan{} server. 
%An installation guide for a server side application of \textan{}.

\subsubsection{Prerequisites}
\label{sssec:SerInstPre}

The \textan{} server requires installation of \emph{Java 8 JRE
\footnote{We recommend to use JRE from \emph{Oracle}, available at: \url{http://www.oracle.com/technetwork/java/javase/downloads/index.html}}}
or a later version and a relational database system (e.g. \emph{MySQL}).
The server should be platform independent, so it runs on any system where JRE is available,
but it depends on native and 3rd party libraries. 
Supported operating systems in the distribution of the \textan{} server are
Linux (32 bit and 64 bit) and Windows (32 bit and 64 bit). \comment[Petr]{Petr}{Finish this...}

\subsubsection{Installation}

The \textan{} server distribution archive contains all files needed for \textan{} server to operate,
such as server binaries, native libraries, run scripts and additional resources.
For installation it is sufficient to unpack its content into any directory.

\subsubsection{Basic configuration}

Before the first start, it is necessary to set up the server and the database.

\comment[Petr,Jakub,Venca]{Petr}{write how to setup server}

\comment[Petr]{Petr}{add some link to \ref{sec:ServerSettings}}

\subsubsection{Starting the server}

The server can be started by starting scripts in its root directory. 
Scripts for different operating systems unfortunately do not provide the same functionality.

The start script for the Linux-based operating system (\emph{run.sh}) runs the server application as daemon
and can be used to run the server as system service.

The start script for the Microsoft Windows OS is less powerful, only runs the server application. 
To run the server as a background process, the Windows Service is required. 
Unfortunately, there are no components shipped with the Jetty Distribution to make it a formal Windows Service.
However, we recommend the use of \emph{Apache ProcRun's Daemon
\footnote{More information can be found on \url{https://commons.apache.org/proper/commons-daemon/procrun.html} and in JavaDoc documentation for \emph{cz.cuni.mff.ufal.textan.server.AppEntry} class.}}.

\subsection{Settings}
\label{sec:ServerSettings}

\subsubsection{Web server settings}

%connector
\paragraph{server.connector.host} The particular interface to listen on. If not set or 0.0.0.0, the web server listens on port on all interfaces.

\paragraph{server.connector.port} The port to listen on. If not set, the web server listens on port 9500.

%thread pool
\paragraph{server.threadPool.maxThreads} The maximum number of threads in web server thread pool. It determines a maximum number of simultaneously opened connections. The default value is 200.

\paragraph{server.threadPool.minThreads} The minimum number of threads in web server thread pool. The default value is 8.

\paragraph{server.threadPool.idleTimeout} The time in milliseconds that the connection can be idle before it is closed.

%ssl
\paragraph{server.ssl}
\paragraph{server.ssl.keyStore.path}
\paragraph{server.ssl.keyStore.password}
\paragraph{server.ssl.keyManager.password}
\paragraph{server.ssl.keyStore.type}
\paragraph{server.ssl.port}

\subsubsection{Database connection settings}
\comment[Venca]{Petr}{Add descrtion for DB properties}
%jdbc
\paragraph{jdbc.driverClassName}
Name of the driver class made usually by database developers which enables to interact with database.

\paragraph{jdbc.url}
 The url that points to our database. The most common url format is like this:
jdbc:[database type]:[hostname]:[port number]/[database name]
and is specific to the driver we use.

\paragraph{jdbc.user}
Username to log into the database.
\paragraph{jdbc.pass}
Password to authorize the user in database.

%c3p0 (http://www.mchange.com/projects/c3p0/#configuration_properties)
\paragraph{c3p0.maxPoolSize}
Maximum number of Connections a pool will maintain at any given time.

\paragraph{c3p0.minPoolSize}
Minimum number of Connections a pool will maintain at any given time.

\paragraph{c3p0.initialPoolSize}
Number of Connections a pool will try to acquire upon startup. Should be between minPoolSize and maxPoolSize.

\paragraph{c3p0.acquireIncrement}
Determines how many connections at a time c3p0 will try to acquire when the pool is exhausted.

\paragraph{c3p0.maxIdleTime}
Seconds a Connection can remain pooled but unused before being discarded. Zero means idle connections never expire.

\paragraph{c3p0.checkoutTimeout}
The number of milliseconds a client calling getConnection() will wait for a Connection to be checked-in or acquired when the pool is exhausted. Zero means wait indefinitely. Setting any positive value will cause the getConnection() call to time-out and break with an SQLException after the specified number of milliseconds.

\paragraph{c3p0.maxStatements}
The size of c3p0's global PreparedStatement cache. If both maxStatements and maxStatementsPerConnection are zero, statement caching will not be enabled. If maxStatements is zero but maxStatementsPerConnection is a non-zero value, statement caching will be enabled, but no global limit will be enforced, only the per-connection maximum. maxStatements controls the total number of Statements cached, for all Connections. If set, it should be a fairly large number, as each pooled Connection requires its own, distinct flock of cached statements. As a guide, consider how many distinct PreparedStatements are used frequently in your application, and multiply that number by maxPoolSize to arrive at an appropriate value. Though maxStatements is the JDBC standard parameter for controlling statement caching, users may find c3p0's alternative maxStatementsPerConnection more intuitive to use. 

\paragraph{c3p0.maxStatementsPerConnection}
The number of PreparedStatements c3p0 will cache for a single pooled Connection. If both maxStatements and maxStatementsPerConnection are zero, statement caching will not be enabled. If maxStatementsPerConnection is zero but maxStatements is a non-zero value, statement caching will be enabled, and a global limit enforced, but otherwise no limit will be set on the number of cached statements for a single Connection. If set, maxStatementsPerConnection should be set to about the number distinct PreparedStatements that are used frequently in your application, plus two or three extra so infrequently statements don't force the more common cached statements to be culled. Though maxStatements is the JDBC standard parameter for controlling statement caching, users may find maxStatementsPerConnection more intuitive to use.

\paragraph{c3p0.idleConnectionTestPeriod}
If this is a number greater than 0, c3p0 will test all idle, pooled but unchecked-out connections, every this number of seconds.

%hibernate
\paragraph{hibernate.dialect}
The classname of a Hibernate org.hibernate.dialect.Dialect which allows to generate SQL optimized for a particular relational database.
\paragraph{hibernate.show\_sql}
If true all sql statements are written to console. It's an alternative for logging.

%NameTag
\subsubsection{Named entity recognizer settings}
This section is mainly about settings stored in NametagLearning.properties.
There are parameters for neural network which nametag use for learning and switches for nametag functions.
Also, there are settings for affecting models storing and learning data generating.
All inputs are assumed to be in UTF-8 encoding.


\paragraph{ner\_identifier}
Identifier of the named entity recognizer type. This affects the tokenizer used in this model, and in theory any other aspect of the recognizer. Supported values are:
\begin{description}
\item[$\cdot$] czech
\item[$\cdot$] english
\item[$\cdot$] generic
\end{description}

\paragraph{tagger}
Identifier of tagger. NameTag can utilize several taggers to obtain the tags and lemmas:

\begin{description}
\item[$\cdot$ trivial] Do not use any tagger. The lemma is the same as the given form and there is no part of speech tag.
\item[$\cdot$ external] Use some external tagger. The input "forms" can contain multiple tab-separated values, first being the form, second the lemma and the rest is part of speech tag. The part of speech tag is optional. The lemma is also optional and if missing, the form itself is used as a lemma.
\item[$\cdot$ morphodita:model] Use MorphoDiTa as a tagger with the specified model. This tagger model is embedded in resulting named entity recognizer model. The lemmatizer model of MorphoDiTa is recommended, because it is very fast, small and detailed part of speech tags do not improve the performance of the named entity recognizer significantly.
\end{description}

\paragraph{featuresFile}
Path of generated file, based on property file 
The recognizer utilizes feature templates to generate features which are used as the input to the named entity classifier. The feature templates are specified in a file, one feature template on a line. Empty lines and lines starting with \# are ignored.

The first space-separated column on a line is the name of the feature template, optionally followed by a slash and a window size. The window size specifies how many adjacent words can observe the feature template value of a given word, with default value of 0 denoting only the word in question.

List of commonly used feature templates follows. Note that it is probably not exhaustive (see the sources in the features directory).

\paragraph{stages}
The number of stages performed during recognition. Common values are either 1 or 2. With more stages, the model is larger and recognition is slower, but more accurate.

\paragraph{iterations}
The number of iterations performed when training each stage of the recognizer. With more iterations, training take longer (the recognition time is unaffected), but the model gets over-trained when too many iterations are used. Values from 10 to 30 or 50 are commonly used.

\paragraph{missing\_weight}
Default value of missing weights in the log-linear model. Common values are small negative real numbers like -0.2.

\paragraph{initial\_learning\_rage}
learning rate used in the first iteration of SGD training method of the log-linear model. Common value is 0.1.

\paragraph{final\_learning\_rage}
learning rate used in the last iteration of SGD training method of the log-linear model. Common values are in range from 0.1 to 0.001, with 0.01 working reasonably well.

\paragraph{gaussian}
The value of Gaussian prior imposed on the weights. In other words, value of L2-norm regularizer. Common value is either 0 for no regularization, or small real number like 0.5.

\paragraph{hidden\_layer}
Experimental support for hidden layer in the artificial neural network classifier. To not use the hidden layer (recommended), use 0. Otherwise, specify the number of neurons in the hidden layer. Please note that non-zero values will create enormous models, slower recognition and are not guaranteed to create models with better accuracy.

\paragraph{heldout\_data}
Optional parameter with heldout data in the described format. If the heldout data is present, the accuracy of the heldout data classification is printed during training. The heldout data is not used in any other way.

\paragraph{waiting\_time}
Maximum time for learning new model (only when learning is blocking - model doesn't exist)

\paragraph{default\_training\_data\_file}
Sample file with initial training data, which is copied before training data generated from database. When this property is empty, no default training data are used.

\paragraph{maximum\_stored\_models}
Number of maximum stored models. When limit is exceeded, the oldest models are deleted

\paragraph{form}
Size of window when forms are used as features (0 when are not used)

\paragraph{lemma}
Size of window when lemma ids are used as features (0 when are not used)

\paragraph{raw\_lemma}
Size of window when raw lemmas are used as features (0 when are not used)

\paragraph{raw\_lemma\_capitalization}
Size of window when capitalization of raw lemma is used as features (0 when are not used)

\paragraph{tag}
Size of window when tags are used as features (0 when are not used)

\paragraph{numeric\_time\_value}
recognize numbers which could represent hours, minutes, hour:minute time, days, months or years

\paragraph{czech\_lemma\_term}
feature template specific for Czech morphological system by Jan Hajič (Hajič 2004). The term information (personal name, geographic name, ...) specified in lemma comment are used as features.

\paragraph{brown\_clusters}size of window foe Brown clusters 

\paragraph{brown\_clusters\_file}
Use Brown clusters found in the specified file. An optional list of lengths of cluster prefixes to be used in addition to the full Brown cluster can be specified. Each line of the Brown clusters file must contain two tab-separated columns, the first of which is the Brown cluster label and the second is a raw lemma.

\paragraph{gazetteers}
Size of window when gazetteers are used (0 when are not used)

\paragraph{gazetteers\_directory}
Directory of gazetteers files. Each file is one gazetteers list independent of the others and must contain a set of lemma sequences, each on a line, represented as raw lemmas separated by spaces.

\paragraph{previous\_stage}
use named entities predicted by previous stage as features

\paragraph{email\_detector}
Detect URLs and emails. If an URL or an email is detected, it is immediately marked with specified named entity type and not used in further processing.

\subsection{Data formats}

\section{Client}

This section contains all information needed to successfully install, configure and run \textan{} client side application.
However thanks to \textan{} client-server architecture and extensive use of webservices,
it is easy to prepare own implementation in any programmer language desired or even use general solutions as SoapUI
\footnote{More information can be found at \url{http://www.soapui.org/}},
if \textan{} default client does not provide all required features, user experience or integration to legacy system is needed.

\subsection{Installation guide}

This section describes how to install and configure \textan{} client on workstations of end users.

\subsubsection{Prerequisites}

The \textan{} client has the same requirement for \emph{Java 8 JRE} or later as the server (see \ref{sssec:SerInstPre}).
However, unlike the server side, it is pure Java application so there are no other additional limitations for operating system etc.
Any platform where JRE 8 is available can support the client.

\subsubsection{Installation}

Unpack an archive with the \textan{} client distribution into any directory. 
The archive contains a .jar file and starting scripts for the \emph{Microsoft Windows} and Linux-based operating systems
\footnote{run.bat for MS Windows and run.sh for Linux clones.\label{runscript_note}}.

\subsubsection{Basic configuration}

If the client is not supposed to connect to the default server, it is necessary to configure a server location. 
The client settings are by default stored in \emph{TextAn.properties\footnote{A description of format of .properties file can be found at \url{http://en.wikipedia.org/wiki/.properties}}}
in the client directory. 
Simply edit or add following lines to properties file (create it if it does not exist),
replacing default server address with the actual one:
\begin{lstlisting}[frame=single,language=properties]
#url of the document processor
url.document=http://localhost:9500/soap/document
#url of the document processor wsdl
url.document.wsdl=http://localhost:9500/soap/document?wsdl
#url of the data provider
url.data=http://localhost:9500/soap/data
#url of the data provider wsdl
url.data.wsdl=http://localhost:9500/soap/data?wsdl
\end{lstlisting}

If SSL is in use, it needs to be configured like this:
\begin{lstlisting}[frame=single,language=properties]
#should ssl be used for communication?
ssl=true
#path to trust store
ssl.trustStore=c\:/temp/clientTrustStore
#trust store password
ssl.trustStore.password=MYPASS
#trust store type (default JKS)
ssl.trustStore.type=JKS
\end{lstlisting}

\subsubsection{Starting the client}

The .jar file from the distribution archive is executable,
but we recommend to use starting scripts\footref{runscript_note}.
Please consult documentation of your Java Platform provider,
if running scripts are not available for your system.
For description of command line arguments see \ref{ssec:CliCmdArg}

\subsubsection{Uninstallation}

The client application does not use any files or resources outside its install directory,
unless explicitly told to do otherwise.
To uninstall the client just delete its installation folder
and all files explicitly used as configuration storage.

\subsection{Settings}

This section describes all settings affecting the client behaviour.
The two basic means are command line arguments and configuration file in properties format.

\subsubsection{Command line arguments}
\label{ssec:CliCmdArg}

The client has one command line option (-h, --help, /H, /?) which displays the usage information.
Apart from that it takes only one command line argument which is the location of settings file.
If no file is specified, default property file \emph{TextAn.properties} will be used.
If '-' is provided, the standard input will be read and settings will not be stored.

\subsubsection{TextAn.properties}

\comment[Adam]{Adam}{move this section as appendix to developer documentation, leave only reference here}
This section contains a list of properties from configuration file that controls
client behaviour and their brief description.

\begin{lstlisting}[frame=single,language=properties]
#main application window height
application.height=600.0
#is main application window maximized?
application.max=true
#main application window width
application.width=800.0
#main application window x-pos
application.x=192.0
#main application window y-pos
application.y=119.0
#indicator whether the filters in context menus in pipeline
#should be cleared when item is selected
clear.filters=true
#document edit/add window height
document.edit.height=300.0
#is document edit/add window maximized?
document.edit.maximized=true
#document edit/add window width
document.edit.width=450.0
#document edit/add window x-pos
document.edit.x=193.0
#document edit/add window y-pos
document.edit.y=164.0
#document view window height
document.viewer.height=576.0
#is document view window height maximized?
document.viewer.maximized=true
#document view window widht
document.viewer.width=662.0
#document view window x-pos
document.viewer.x=148.0
#document view window y-pos
document.viewer.y=203.0
#number of documents to list on one page in document
#view window
documents.per.page=100
#document list window height
documents.viewer.height=604.0
#is document list window maximized?
documents.viewer.maximized=true
#document list window width
documents.viewer.width=663.0
#document list window x-pos
documents.viewer.x=513.0
#document list window y-pos
documents.viewer.y=87.0
#default distance from graph center to display
graph.distance=6
#graph window height
graph.viewer.height=784.0
#is graph window maximized?
graph.viewer.maximized=true
#graph window width
graph.viewer.width=678.0
#graph window x-pos
graph.viewer.x=60.0
#graph window y-pos
graph.viewer.y=26.0
#flag indicating whether the hypergraphs should be displayed
#as background color instead of by additional vertex
hypergraphs=true
#object join window height
join.view.height=682.0
#is object join window maximized?
join.view.maximized=true
#object join window width
join.view.width=953.0
#object join window x-pos
join.view.x=147.0
#object join window y-pos
join.view.y=91.0
#directory lastly used for save/load a report
loadreport.dir=C\:\\temp
#application language
locale.language=cs
#number of documents to list on left page
#in object join window
objects.per.page.left=100
#number of documents to list on right page
#in object join window
objects.per.page.right=100
#number of documents to list on one page
#in object list window
objects.per.page=100
#relation list window height
relation.view.height=741.0
#is relation list window maximized
relation.view.maximized=true
#relation list window width
relation.view.width=722.0
#relation list window x-pos
relation.view.x=149.0
#relation list window y-pos
relation.view.y=15.0
#report wizard pipeline window height
report.wizard.height=734.0
#is report wizard pipeline window maximized?
report.wizard.maximized=true
#report wizard pipeline window width
report.wizard.width=843.0
#report wizard pipeline window x-pos
report.wizard.x=397.0
#report wizard pipeline window y-pos
report.wizard.y=70.0
#report wizard pipeline window height
selectfile.dir=C\:\\temp
#should ssl be used for communication?
ssl=true
#path to trust store
ssl.trustStore=c:/Users/System_Lords/Documents/NetBeansProjects/TextAn/Client/myTrustStore.localhost
#trust store password
ssl.trustStore.password=TextAn2014KS
#trust store type (default JKS)
ssl.trustStore.type=JKS
#files with extension 'txt' should be processed as text files
#encoded in Windows-1250 encoding only other valid value for
#now is TEXT_UTF8 for utf-8 encoding
selectfile.extension.txt.type=TEXT_CP1250
#url of the data provider wsdl
url.data.wsdl=http\://localhost\:9500/soap/data?wsdl
#url of the data provider
url.data=http\://localhost\:9500/soap/data
#url of the document processor wsdl
url.document.wsdl=http\://localhost\:9500/soap/document?wsdl
#url of the document processor
url.document=http\://localhost\:9500/soap/document
#user's login
username=BFU
#flag indicating whether independent system windows should
#be used instead of embedded inner windows
windows.independent=false
\end{lstlisting}

For more information about \emph{clear.filters}, \emph{graph.distance},
\emph{hypergraphs}, \emph{locale.language}, \emph{username}
and \emph{windows.independent} please see \ref{ssec:Settings}

\end{document}
